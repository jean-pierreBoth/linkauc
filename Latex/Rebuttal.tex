%%%%%%%%%%%%%%%%%%%%%%%%%%%%%%%%%%%%%%%%%%%%%%%%%%%%%%%%%%%%%%%
%
%Link AUC rebuttal%
%%%%%%%%%%%%%%%%%%%%%%%%%%%%%%%%%%%%%%%%%%%%%%%%%%%%%%%%%%%%%%%
\documentclass{article}
\usepackage{xcolor}
\usepackage{hyperref}
\usepackage{authblk}
\usepackage{numprint}
\usepackage{siunitx}
%------------------------------------------------------------------------------
%  syles defined in linkauc.sty
\usepackage{625333_0_supp_latex_10774581_sbxt8w}
% ----------------------------------------------------------------------

\bibliographystyle{pnas2009}

\title{Response to reviewers' comments}
\author[1]{Jianshu Zhao}
\author[2,*]{Jean, Pierre Both}
\affil[1]{Center for Bioinformatics and Computational Genomics, Georgia Institute of Technology, Atlanta, Georgia, USA}
\affil[2]{Université Paris-Saclay, CEA, List, Palaiseau, France. (Retired)}
\affil[*]{Corresponding author : jeanpierre.both@gmail.com} 
\date{June 2024}
\begin{document}
\maketitle

\textbf{Comment}: Menand and Seshradhi have noted that your Letter assumes that $d_i$ 
is the degree of the original graph, but actually $d_i$ is the degree 
of node i in the test graph and thus exactly the number of removed 
edges incident to vertex i. They do not measure/consider the training 
edges in the measurement, either in the numerator or denominator of the VCMPR. 
The authors concede that this definition is not as clear in the text as it could be.\\

\textcolor{blue}{\textbf{Response}: We want to thank the authors for pointing out that our definition for VCMPR@k is not clear.
We now make it clear in the main text that $d_i$ is the degree 
of node i in the test graph (the number of removed 
edges incident to vertex i), but not the orginal graph. The changes are also shown below:}\\

\textcolor{red}{$VCMPR@k$ is defined as the mean over nodes i of  $ VCMPR_{i}(k)= \frac{t_{i}(k)}{\min(r_{i},k)}$ with $ r_{i} $ is the number of edges deleted from the full graph to the train graph. For many real-world graphs, nodes degrees follow a power law distribution, a large fraction of nodes may have degree much lower than the mean degree.
This impacts the quantity $ VCMPR_{i}(k)$ as a node with no edge deleted cannot contribute to $VCMPR@k$ for a given deletion probability $p$.} \\


\textcolor{blue}{Accordingly, a large fraction of nodes may have degree much lower than the mean degree.
This impacts $VCMPR@k$  as $VCMPR_{i}(k)$ is undefined for nodes with no edge deleted. Our software \href{https://github.com/jean-pierreBoth/graphembed}{\color{blue}graphembed} was also modified to compute $VCMPR@k$ based on test graph (deleted edges). 
You can run the software following the code example \href{https://github.com/jean-pierreBoth/linkauc/blob/master/running_graphembed.md}{\color{blue}here}.Updated results in our original version is also attached:} \\

\textcolor{red}{
\begin{tabular*}{\textwidth}[]{p{1.8cm}@{\extracolsep\fill}ccccccc}
    \toprule
    Graph &  Algo Parameters &VCMPR@10 & Centric AUC &  Global AUC  \\
    \midrule
    \multicolumn{5}{l}{A. Sketching results. Parameters given in a triplet: dimension, nbhops, decay}\\
    Blog   & 1000,5,0.5 & $0.0437 \pm \num{3.1E-4}$ & $0.681 \pm \num{5.9E-3}$ & $0.93 \pm \num{1.2E-2}$ \\
    Amazon & 1000,5,0.5 & $0.5418 \pm \num{1.3E-2}$ & $0.961 \pm \num{4E-3}$    & $ 0.978 \pm \num{3.2E-4}$ \\
    Dblp   & 1000,5,0.5 & $0.5713 \num{1.35E-2}$ & $0.9048 \pm \num{6.5E-3}$    & $ 0.9148 \pm \num{6.6E-4}$ \\
    Dblp   & 400,4,0.4  & $0.5882 \pm \num{1.31E-2}$ & $0.94 \pm \num{5.1E-3}$   & $ 0.961 \pm \num{4.4E-4}$ \\
    \midrule
    \multicolumn{5}{l}{B. Hope embedding. Parameters given in a couple: dimension, nb iterations}\\
    Blog   & 400,5   & $0.1669 \pm \num{5.1E-3}$   & $0.698 \pm \num{7.4E-3}$  & 0.952 \\
    Amazon & 400,5   & $0.04544 \pm \num{2.9E-4} $  & $0.834 \pm \numprint{6.4E-3}$    & $ 0.856 \pm \num{9.1E-4}$ \\
    \bottomrule
\end{tabular*}
}

\bibliography{625333_0_supp_latex_10774582_sbxt8w}
\end{document}