%%%%%%%%%%%%%%%%%%%%%%%%%%%%%%%%%%%%%%%%%%%%%%%%%%%%%%%%%%%%%%%
%
%Link AUC manuscript%
%%%%%%%%%%%%%%%%%%%%%%%%%%%%%%%%%%%%%%%%%%%%%%%%%%%%%%%%%%%%%%%
\documentclass{article}
\usepackage{xcolor}
\usepackage{hyperref}
\usepackage{authblk}
\usepackage{numprint}
\usepackage{siunitx}
%------------------------------------------------------------------------------
%  syles defined in linkauc.sty
\usepackage{625333_0_supp_latex_10774581_sbxt8w}
% ----------------------------------------------------------------------


\bibliographystyle{pnas2009}

\title{Response to reviewers's comments}

\date{June 2024}

\begin{document}

\maketitle

Menand and Seshradhi have noted that your Letter assumes that d_i 
is the degree of the original graph, but actually d_i is the degree 
of node i in the test graph and thus exactly the number of removed 
edges incident to vertex i. They do not measure/consider the training 
edges in the measurement, either in the numerator or denominator of the VCMPR. 
The authors concede that this definition is not as clear in the text as it could be.

\textbf{Response}: We want to thank the authors for pointed out that our definition for VCMPR is not clear.
We now make it clear in the main text that $d_i$ is the degree 
of node i in the test graph (the number of removed 
edges incident to vertex i), but not the orginal graph. 


\bibliography{625333_0_supp_latex_10774582_sbxt8w}
\end{document}